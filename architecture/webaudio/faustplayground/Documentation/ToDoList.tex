
\documentclass[a4paper]{article}
\usepackage{pslatex}
\usepackage[T1]{fontenc}
\usepackage[utf8x]{inputenc}
\setlength\parskip{\medskipamount}
\setlength\parindent{0pt}
\usepackage{graphicx}
\usepackage{amssymb}
\usepackage{lscape}
%\usepackage{hyperref}

\usepackage{multicol}

\makeatletter

\providecommand{\boldsymbol}[1]{\mbox{\boldmath $#1$}}
\newcommand{\ASL}			{ASL}
\newcommand{\OSC}[1]		{\texttt{#1}}
\newcommand{\lra}			{$\leftrightarrow$}
\newcommand{\seg}[1]		{Seg(#1)}

\setlength{\parskip}{1mm}

\makeatother

\begin{document}

\title{TO DO ON FAUSTPLAYGROUND \\ v.1.0}

\author{Grame, Centre National de Création Musicale\\
{\small <research@grame.fr>} \\
\vspace{2mm}
[ANR-12-CORD- 0009] and [ANR-13-BS02-0008]
}

\maketitle

\topskip0pt

\newpage
\section{La Forme}

\begin{itemize}
\item Le resize doit toujours bien se passer (notamment pour le connexions)
\item La bibliothèque ne se ferme pas toujours très bien selon comment on en sort
\end{itemize}

\section{Les Fonctionnalités}
\subsection{En Interne}
\begin{itemize}
\item Save/Recall des scènes (trouver le bon moyen d'enregistrer la session en ligne). 
Les pistes :
\begin{itemize}				
\item Google Docs
\item Compte Facebook
\item Téléchargement d'un fichier
\item DropBox
\item Email
\item Compte sur le serveur
\end{itemize}
\item Pouvoir faire resize sur les modules et avoir une scrollbar
\item Pouvoir réduire la taille des modules (style itunes passage en mode mini-player)
\item Ajouter un bouton de bypass sur les modules
\item mieux gérer les erreurs (compilation et faustweb)
\item Assurer la compatibilité des plateformes (notamment Safari et l'export)
\item Peut-être modifier la manière de déconnecter des modules (ce n'est pas évident pour tout le monde)
\item Assurer que ça se passe bien quand il y a plusieurs instances du même effet et qu'on calcule le faust equivalent
\end{itemize}

\subsection{Demande du groupe TICE des profs de musique}
\begin{itemize}
\item interface des instruments plus "standardisés" qu'on comprenne mieux comment en jouer
\item mettre le nom des notes plutôt que des int sur les instruments type harpe
\item pouvoir enregistrer les scènes
\item pouvoir partager les scènes enregistrées
\item choix du logo de l'application téléchargée sur le téléphone
\item pouvoir connaitre les modules qui ont été utilisés à partir d'une application finie ou forcer une capture d'écran avec l'export...
\item pouvoir choisir l'axe sur lequel on va jouer d'un paramètre une fois l'appli compilée
\item penser au déploiement sur tablette
\end{itemize}

\subsection{Demande de Philippe Jeanjacquot de l'institut national de l'éducation}
\begin{itemize}
\item mise à disposition d'application "instruments" pour Android (harpe, sitar, organ, etc)
\item mettre le nom des notes plutôt que des int sur les instruments type harpe
\end{itemize}

\section{Le Déploiement}

\begin{itemize}
\item Ajouter un lien vers les playgrounds depuis le site de Faust
\item Mettre une bibliothèque jolie des effets publiés dans www/library sur le site de Faust
\end{itemize}

\section{Les choses sombres et dures}

En dur dans le serveur, on trouve le chemin d'accès aux ressources de la bibliothèque ==> http://faust.grame.fr/www/libfaust.js par exemple. Ce serait peut-être bien de choisir une structure de serveur définitive pour mettre les chemins en relatifs. Par exemple avoir un dossier js avec à la fois les fichies liés à fausptlayground et ceux liés à libfaust. \\

Les endroits avec des chemins d'accès à modifier sont :
\begin{itemize}
\item index.html
\item pedagogie/index.html
\item js/Pedagogie/Library.js
\item js/Scenes/Finish.js
\end{itemize}
\end{document}




